\documentclass{beamer}
\usepackage[utf8]{inputenc}
\usepackage[portuges,brazilian]{babel}
\usepackage[T1]{fontenc}
\usetheme{Warsaw}
\title[Performance em Python]{O Fim está próximo, Onde você pretende passar sua eternidade computacional?}
\author{Marcelo A Caetano -- Titans Group}
\date{Dec 21, 2012 -- aka the end of the world as we know}

\begin{document}
  \begin{frame}
    \titlepage
  \end{frame}
  \begin{frame}{Milhões de Instruções por Segundo MIPS}
    \begin{block}{Uma eternidade computacional}
      \begin{enumerate}
        \item<1-| alert@1> De acordo com a Lista Top15 Flops, o Core i7 980-X Extreme Edition da Intel atinge até 20 Gflop/s.
        \item<2 -> isto é, 2 bilhões de operações de ponto flutuante em 1 único milisegundo.
        \item<3 -> Que Bom! Só que... eu não faço muitas operações de ponto flutuante.
        \item<4 -> Ok, ainda assim, seu computador é capaz de executar cerca de 40 mil instruções genéricas, sequenciais, num único milisegundo.
        \item<5 -> Só isso? mas ainda é muito pouco!
      \end{enumerate}
    \end{block}
  \end{frame} 
  \begin{frame}{Algorítmos}
    \begin{enumerate}
     \item<1 -> Algorítmos são estruturas matemáticas utilizadas para solução de problemas repetitivos, tomadas de decisões e etc.
     \item<2 -> Aprenda a usar os algoritmos e entender sua complexidade computacional.
     \item<3 -> É possível, com o algoritmo certo, reduzir a quantidade de buscas de milhões para dezenas.
    \end{enumerate}
  \end{frame}
  \begin{frame}{Exemplo Prático}
   
  \end{frame}

\end{document}
